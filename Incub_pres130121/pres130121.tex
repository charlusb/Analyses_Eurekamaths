%%%%%%%%%%%%%%%%%%%%%%%%%%%%%%
% Modèle type de présentation
%%%%%%%%%%%%%%%%%%%%%%%%%%%%%%

%-----------
% Préambule
%-----------

% faire tourner avec xelatex file.tex

% Type de document et encodage :
\documentclass[11pt]{beamer}
\usepackage[utf8x]{inputenc}
\usetheme[progressbar=foot,numbering=none]{metropolis}
%% \hypersetup{pdfpagemode=FullScreen}

% Langue :
\usepackage[french]{babel}

% Mise en forme :
\setbeamertemplate{caption}{\raggedright\insertcaption\par} % figure
\usepackage{enumitem} % listes
\usepackage{graphicx} % images
\usepackage{media9} % videos
\definecolor{foret}{rgb}{0.0353,0.3216,0.1569} % vert forêt
\setbeamercolor{title separator}{fg=celadon} % couleur séparateur de titre
\setbeamercolor{progress bar}{fg=celadon} % couleur barre de progression
 %\definecolor{MyBackground}{RGB}{255,241,167}
\usepackage{animate} %need the animate.sty file
\setbeamercolor{background canvas}{bg=white}
\definecolor{darkterracotta}{rgb}{0.8, 0.31, 0.36}
\definecolor{aquamarine}{rgb}{0.5, 1.0, 0.83}
\definecolor{carminepink}{rgb}{0.92, 0.3, 0.26}
\definecolor{burntsienna}{rgb}{0.91, 0.45, 0.32}
\definecolor{celadon}{rgb}{0.67, 0.88, 0.69}
\definecolor{caribbeangreen}{rgb}{0.0, 0.8, 0.6}
\definecolor{bittersweet}{rgb}{1.0, 0.44, 0.37}
\setbeamercolor{normal text}{fg=black, bg=black}
\newcommand{\ig}{\includegraphics}
\makeatletter
\setlength{\metropolis@titleseparator@linewidth}{1pt} % épaisseur séparateur de titre
\setlength{\metropolis@progressonsectionpage@linewidth}{2pt} % épaisseur barre de progression (section)
\setlength{\metropolis@progressinheadfoot@linewidth}{3pt} % épaisseur barre de progression (bas de page)
\makeatother

	
                  
\begin{document}


\title{\includegraphics[scale=0.10]{logoincc.jpg}  \includegraphics[scale=0.3]{ed3c.png} \includegraphics[scale=0.05]{logouniv.jpg}  \includegraphics[scale=0.03]{logocnrs.png} \\
      
  {\textcolor{black}{Eureka moments in the acquisition of mathematical concepts}}\\
  {\small {\textcolor{black}{Charlotte Barot, Louise Chevalier, Lucie Martin and  V\'{e}ronique Izard }}}\\
        
     }

  
        \begin{frame}
          \titlepage
        \end{frame}



        \section{Introduction}

        \begin{frame}
          
          How does it feel to understand a mathematical concept ?

        \end{frame}


        

             
        \begin{frame}
          Famous scientists experience sudden \textcolor{bittersweet}{``Eureka moments''}, notably in mathematics \footnotesize{Poincar\'{e} 1908, Hadamard 1959}

          \normalsize
          
          \textcolor{white}{Sudden and unexpected understanding}

          \textcolor{white}{Feeling of certainty}


          \textcolor{white}{Similar experiences are studied in the scope of problem solving where solutions come by ``insight''}

          \textcolor{white}{No awareness that it was about to come  \footnotesize{Metcalfe 1986, Metcalfe and Wiebe 1987, Bowden and Jung-Beeman 2003}}

          \normalsize

          \textcolor{white}{Immediately perceived as correct and relevant \footnotesize{Kounios and Beeman 2014, Danek and Wiley 2017, Laukkonen et al. 2020}}
 

        \end{frame}

   
                
        \begin{frame}
          Famous scientists experience sudden \textcolor{bittersweet}{``Eureka moments''}, notably in mathematics \footnotesize{Poincar\'{e} 1908, Hadamard 1959}


          \normalsize
          
          \textcolor{bittersweet}{Sudden and unexpected understanding}

          \textcolor{white}{Feeling of certainty}


        \textcolor{white}{Similar experiences are studied in the scope of problem solving where solutions come by ``insight''}

        \textcolor{white}{No awareness that it was about to come  \footnotesize{Metcalfe 1986, Metcalfe and Wiebe 1987, Bowden and Jung-Beeman 2003} }

        \normalsize

          \textcolor{white}{Immediately perceived as correct and relevant \footnotesize{Kounios and Beeman 2014, Danek and Wiley 2017, Laukkonen et al. 2020}}
 

        \end{frame}

               


        \begin{frame}

          Famous scientists experience sudden \textcolor{bittersweet}{``Eureka moments''}, notably in mathematics \footnotesize{Poincar\'{e} 1908, Hadamard 1959}



          \normalsize
          
          \textcolor{bittersweet}{Sudden and unexpected understanding}

          \textcolor{bittersweet}{Feeling of certainty}


          \textcolor{white}{Similar experiences are studied in the scope of problem solving where solutions come by ``insight''}

          \textcolor{white}{No awareness that it was about to come  \footnotesize{Metcalfe 1986, Metcalfe and Wiebe 1987, Bowden and Jung-Beeman 2003}}

          \normalsize

          \textcolor{white}{Immediately perceived as correct and relevant \footnotesize{Kounios and Beeman 2014, Danek and Wiley 2017, Laukkonen et al. 2020}}
 


          %% \centering

          %% \includegraphics[scale=0.05]{doty.png}

          %% \footnotesize{The nine dot problem, Lung and  Dominowski 1985}

        \end{frame}

        \begin{frame}

          Famous scientists experience sudden \textcolor{bittersweet}{``Eureka moments''}, notably in mathematics \footnotesize{Poincar\'{e} 1908, Hadamard 1959}

          \normalsize
          
          \textcolor{bittersweet}{Sudden and unexpected understanding}

          \textcolor{bittersweet}{Feeling of certainty}


            Similar experiences are studied in the scope of problem solving where solutions come by ``insight''

            \textcolor{white}{No awareness that it was about to come \footnotesize{Metcalfe 1986, Metcalfe and Wiebe 1987, Bowden and Jung-Beeman 2003}}

            \normalsize

          \textcolor{white}{Immediately perceived as correct and relevant \footnotesize{Kounios and Beeman 2014, Danek and Wiley 2017, Laukkonen et al. 2020} }
 


          %% \centering

          %% \includegraphics[scale=0.05]{doty.png}

          %% \footnotesize{The nine dot problem, Lung and  Dominowski 1985}

        \end{frame}
        
         \begin{frame}

          Famous scientists experience sudden \textcolor{bittersweet}{``Eureka moments''}, notably in mathematics \footnotesize{Poincar\'{e} 1908, Hadamard 1959}

          \normalsize
          
          \textcolor{bittersweet}{Sudden and unexpected understanding}

          \textcolor{bittersweet}{Feeling of certainty}


            Similar experiences are studied in the scope of problem solving where solutions come by ``insight''

          \textcolor{bittersweet}{No awareness that it was about to come  \textcolor{black}{\footnotesize{Metcalfe 1986, Metcalfe and Wiebe 1987, Bowden and Jung-Beeman 2003}}}

          \normalsize

          \textcolor{white}{Immediately perceived as correct and relevant \footnotesize{Kounios and Beeman 2014, Danek and Wiley 2017, Laukkonen et al. 2020} }
 


          %% \centering

          %% \includegraphics[scale=0.05]{doty.png}

          %% \footnotesize{The nine dot problem, Lung and  Dominowski 1985}

        \end{frame}

 \begin{frame}

          Famous scientists experience sudden \textcolor{bittersweet}{``Eureka moments''}, notably in mathematics \footnotesize{Poincar\'{e} 1908, Hadamard 1959}

          \normalsize

          \textcolor{bittersweet}{Sudden and unexpected understanding}

          \textcolor{bittersweet}{Feeling of certainty}


            Similar experiences are studied in the scope of problem solving where solutions come by ``insight''

            \textcolor{bittersweet}{No awareness that it was about to come \textcolor{black}{\footnotesize{Metcalfe 1986, Metcalfe and Wiebe 1987, Bowden and Jung-Beeman 2003}}}

            \normalsize

          \textcolor{bittersweet}{Immediately perceived as correct and relevant \textcolor{black}{\footnotesize{Kounios and Beeman 2014, Danek and Wiley 2017, Laukkonen et al. 2020}}}
 


          %% \centering

          %% \includegraphics[scale=0.05]{doty.png}

          %% \footnotesize{The nine dot problem, Lung and  Dominowski 1985}

        \end{frame}

         
        %%   \begin{frame}

        %%   Famous scientists experience sudden \textcolor{bittersweet}{``Eureka moments''}, notably in mathematics (Poincar\'{e} 1908, Hadamard 1959)

        %%   \textcolor{bittersweet}{Sudden and unexpected understanding}

        %%   \textcolor{bittersweet}{Feeling of certainty}


        %%     Similar experiences are studied in the scope of problem solving where solutions come by ``insight''

        %%   \textcolor{bittersweet}{No awareness that it was about to come }

        %%   \textcolor{bittersweet}{Immediately perceived as correct and relevant }
 


        %%   \centering

        %%   \includegraphics[scale=0.05]{doty.png}

        %%   \footnotesize{The nine dot problem, Lung and  Dominowski 1985}

        %% \end{frame}




 \begin{frame}

   \centering


    \footnotesize{The nine dot problem, Lung and  Dominowski 1985}
   
   \ig[scale=0.4]{nine.png}


   \end{frame}
        \begin{frame}
          \centering
          
          \ig[scale=0.10]{solution.png}

         
          
        \end{frame}


        \subsection{Main question}           
                  


        
                \begin{frame}

                  So far, no evidence of insights while learning a new  concept
                  \bigskip

                  \centering
                  \includegraphics[scale=0.8]{maggie_maths.png}
                  
                \end{frame}

                
                \begin{frame}
                  
                  Methodological difficulties: learning science concepts is difficult and protracted \footnotesize{Carey 2009, Weber 2002, Asmuth and Rips 2006, Vosniadou 2019}

                  \normalsize

                  \begin{itemize}

                    
                  \item{ \textcolor{white}{Conducted in classrooms : cross sectional designs or longitudinal designs with long delays between sessions \footnotesize{Behr et al. Lesh 1984,  Siegler 1995,  Schauble 1996, Rittle-Johnson and Alibali 1999, Smith et al. 2005}}}
                   
                  \normalsize

                \item{ \textcolor{white}{Lab experiment with simple learning target for example inferring a rule for categorizing images \footnotesize{Shepard 1961, Feldman 2000, Goodman et al. 2008, Ohlsson and Cosejo 2014, Marti et al. 2018}}}

                  \end{itemize}
                  
                \end{frame}

                \begin{frame}
                  
                  Methodological difficulties: learning science concepts is difficult and protracted \footnotesize{Carey 2009, Weber 2002, Asmuth and Rips 2006, Vosniadou 2019}

                  \normalsize

                  \begin{itemize}

                    
                 \item{ Conducted in classrooms : cross sectional designs or longitudinal designs with long delays between sessions
                   \footnotesize{Behr et al. Lesh 1984,  Siegler 1995,  Schauble 1996, Rittle-Johnson and Alibali 1999, Smith et al. 2005}}
                   
                  \normalsize

                \item{ \textcolor{white}{Lab experiment with simple learning target for example inferring a rule for categorizing images \footnotesize{Shepard 1961, Feldman 2000, Goodman et al. 2008, Ohlsson and Cosejo 2014, Marti et al. 2018}}}

                  \end{itemize}
                  
                \end{frame}


                \begin{frame}
                  
                  Methodological difficulties: learning science concepts is difficult and protracted \footnotesize{Carey 2009, Weber 2002, Asmuth and Rips 2006, Vosniadou 2019}

                  \normalsize

                  \begin{itemize}

                    
                 \item{ Conducted in classrooms : cross sectional designs or longitudinal designs with long delays between sessions
                   \footnotesize{Behr et al. Lesh 1984,  Siegler 1995,  Schauble 1996, Rittle-Johnson and Alibali 1999, Smith et al. 2005}}
                   
                  \normalsize

                 \item{ Lab experiment with simple learning target for example inferring a rule for categorizing images \footnotesize{Shepard 1961, Feldman 2000, Goodman et al. 2008, Ohlsson and Cosejo 2014, Marti et al. 2018}}

                  \end{itemize}
                  
                \end{frame}


                \section{Goals of the study}

                
                \begin{frame}

                  %% First, we tested whether participants’ objective performance varied as a function of the number of lessons studied. Observing effects of the teaching condition would show that our manipulation was successful in inducing learning, a sine-qua-non requisite for our study. Second, we looked for evidence that participants may have experienced insights, and tested whether these experiences were associated with a better level of objective understanding. Third and lastly, we tested whether insight reports were related to participant’s ratings of confidence, and analyzed how insight reports and ratings of confidence related to performance in the different subtests. Finding a double dissociation in this analysis would indicate that the mechanisms giving rise to insights are at least partially dissociated from the mechanisms informing people’s introspective judgments, supporting the hypothesis that some conceptual learning processes operate covertly and remain inaccessible to introspection.


                  
                  Conceive a one session paradigm of conceptual learning

                 \textcolor{white}{ See if participants report experiencing insights while learning a new concept and if these insights are associated with a better learning}

 
                 


                \end{frame}


                 \begin{frame}



                  
                  Conceive a one session paradigm of conceptual learning

                  See if participants report experiencing insights while learning a new concept and if these insights are associated with a better learning


                 


                 \end{frame}

               


                 \begin{frame}

                   \textcolor{white}{ Do insight rely on mechanisms that are not accessible to conscious reports ? \footnotesize{Schooler et al. 1993,  Jameson et al. 1990, Miner and Reder 1994, Schwartz and Metcalfe, 1992}}

                   \normalsize

                   \textcolor{white}{ Test the accessibility of concepts learning mechanisms}

                   \textcolor{white}{Collect ratings of confidence about participants' knowledge of the new concept and see if insight are related to some specific aspects of performance, even when we control for confidence ratings}
  

                 \end{frame}


                 
                 \begin{frame}

                   Could concept learning rely on mechanisms that are not accessible to conscious reports ? \footnotesize{Schooler et al. 1993,  Jameson et al. 1990, Miner and Reder 1994, Schwartz and Metcalfe, 1992}

                   \normalsize

                   \textcolor{white}{Test the accessibility of concepts learning mechanisms}
                   
                   \textcolor{white}{Collect ratings of confidence about participants' knowledge of the new concept and see if insight are related to some specific aspects of performance, even when we control for confidence ratings}
  
                 
                 \end{frame}


                 \begin{frame}

                   Could concept learning rely on mechanisms that are not accessible to conscious reports ? \footnotesize{Schooler et al. 1993,  Jameson et al. 1990, Miner and Reder 1994, Schwartz and Metcalfe, 1992}


                   \normalsize
                   
                   Test the accessibility of concepts learning mechanisms

                   \textcolor{white}{Collect ratings of confidence about participants' knowledge of the new concept and see if insight are related to some specific aspects of performance, even when we control for confidence ratings}
  
                   %% If the mechanisms giving rise to insights are inaccessible to introspection, we predicted that some learning achievements may be uniquely related to experiencing insights, after factoring out variations in participants’ introspective ratings of understanding

                 \end{frame}


                 

                 \begin{frame}

                   Could concept learning rely on mechanisms that are not accessible to conscious reports ? \footnotesize{Schooler et al. 1993,  Jameson et al. 1990, Miner and Reder 1994, Schwartz and Metcalfe, 1992}
                   \normalsize

                   Test the accessibility of concepts learning mechanisms

                   Collect ratings of confidence about participants' knowledge of the new concept and see if insight are related to some specific aspects of performance, even when we control for confidence ratings  
  
                   %% If the mechanisms giving rise to insights are inaccessible to introspection, we predicted that some learning achievements may be uniquely related to experiencing insights, after factoring out variations in participants’ introspective ratings of understanding

                 \end{frame}

  
                 
                \section{Learning situation}




                \begin{frame}

                  


                  The new concept : geodesic
                  
                  Geodesic is the generalization of straight line to curved surfaces

                  Starting from a given point on a surface, it is a path which has a constant direction and never turns

                \end{frame}
                


                \begin{frame}

                  Learning situation : geodesic on the sphere

                  \centering
                  "Is it straight?"

                  \begin{tabular}{cccc}
  
                    \ig[scale=0.2]{co.png} & \ig[scale=0.2]{gc.png} & \ig[scale=0.2]{pc.png} \\

                    &   &   \\

                  \end{tabular}

                \end{frame}



                \begin{frame}

                  Learning situation : geodesic on the sphere
 
                  \centering
                  "Is it straight?"
                  
                  \begin{tabular}{cccc}
  
                    \ig[scale=0.2]{co.png} & \ig[scale=0.2]{gc.png} & \ig[scale=0.2]{pc.png} \\

                    \ig[scale=0.4]{nein.png} &   &   \\

                  \end{tabular}

                \end{frame}



                \begin{frame}

                  Learning situation : geodesic on the sphere

                  \centering
                  "Is it straight?"

                  \begin{tabular}{cccc}
  
                    \ig[scale=0.2]{co.png} & \ig[scale=0.2]{gc.png} & \ig[scale=0.2]{pc.png} \\

                    \ig[scale=0.4]{nein.png} &  \ig[scale=0.4]{checky.png} &   \\

                  \end{tabular}

                  \end{frame}
                


                \begin{frame}


                  Learning situation : geodesic on the sphere



                  \centering
                  "Is it straight?"

                  \begin{tabular}{cccc}
  
                    \ig[scale=0.2]{co.png} & \ig[scale=0.2]{gc.png} & \ig[scale=0.2]{pc.png} \\

                    \ig[scale=0.4]{nein.png} &  \ig[scale=0.4]{checky.png} &   \ig[scale=0.4]{nein.png} \\

                  \end{tabular}

                \end{frame}


                \begin{frame}

                  \ig[scale=0.85]{learningy.png}

                  \centering
                  The rubber band lesson

                  \ig[scale=0.16]{elastoc.jpg}  \ig[scale=0.04]{elastoc1.jpg}  \ig[scale=0.04]{elastoc2.jpg}

                  \tiny{ The elastic is straight and follows greater circles on
                    a sphere, but when applied to a small circle, the elastic does not follow the line }

                  \bigskip

                  Four conditions : 1, 3, 5 or 7 different lessons

                \end{frame}

                \begin{frame}

                  \ig[scale=0.85]{learningy.png}

                  \centering
                  The rubber band lesson

                  \ig[scale=0.16]{elastoc.jpg}  \ig[scale=0.04]{elastoc1.jpg}  \ig[scale=0.04]{elastoc2.jpg}

                  \tiny{ The elastic is straight and follows greater circles on
                    a sphere, but when applied to a small circle, the elastic does not follow the line }

                  \bigskip

                  Four conditions : 1, 3, 5 or 7 different lessons

                  Goal : test whether participants’ objective performance vary as a function of the number of lessons studied



                  
                \end{frame}

                


                \begin{frame}
                  
                  
                  \centering
                  \ig[scale=0.99]{protoc.png}

                  \tiny{N=56, 18-43 years (\it M=25.5),
                    time $ \thickapprox $ 1h30}
                  
                \end{frame}

                \begin{frame}{Preliminary tests}

                  \ig[scale=0.27]{pretesty.pdf}

                  \centering

                  \ig[scale=0.22]{ligne.png}

                  \centering
                  \ig[scale=0.02]{cot}  \ig[scale=0.02]{gct}   \ig[scale=0.02]{pct}


                \end{frame}


                  \section{Performance measures}

                  \begin{frame}

                    \ig[scale=0.85]{test1y.png}
  
                    \centering

                    \begin{tabular}{cccc}

                      \ig[scale=0.02]{cot} &   \ig[scale=0.02]{gct} &  \ig[scale=0.02]{pct}  \\

                      \tiny{Not straight non planar lines} & \tiny {Straight planar lines} & \tiny{Not straight planar lines} \\
                      
                    \end{tabular}

  
                  \end{frame}



                  \begin{frame}

                    \ig[scale=0.85]{protoct2.png}

                    \centering

                    \begin{tabular}{cccc}

                      \ig[scale=0.3]{conenf.png} &   \ig[scale=0.3]{coneof.png} &  \ig[scale=0.3]{conend.png} &  \ig[scale=0.3]{coneod} \\
                      &  &  &   \\

                      &   &  &  \\

                    \end{tabular}

                    

                  \end{frame}



                  \begin{frame}

                    \ig[scale=0.85]{protoct2.png}
                    
                    \centering

                    \begin{tabular}{cccc}

                      \ig[scale=0.3]{conenf.png} &   \ig[scale=0.3]{coneof.png} &  \ig[scale=0.3]{conend.png} &  \ig[scale=0.3]{coneod.png} \\

                      \ig[scale=0.2]{nein.png} &  &  &   \\

                      \tiny{Non straight non planar lines} &   &  &  \\

                    \end{tabular}

                  
                  \end{frame}


                  \begin{frame}

                    \ig[scale=0.85]{protoct2.png}

                    \centering

                    \begin{tabular}{cccc}

                      \ig[scale=0.3]{conenf.png} &   \ig[scale=0.3]{coneof.png} &  \ig[scale=0.3]{conend.png} &  \ig[scale=0.3]{coneod.png} \\
                      
                      \ig[scale=0.2]{nein.png} &  \ig[scale=0.2]{checky.png} &  &   \\

                      \tiny{Non straight non planar lines} &  \tiny{Straight planar lines} &  &  \\
                    \end{tabular}

                    \end{frame}


                    \begin{frame}
                      
                      
                      \ig[scale=0.85]{protoct2.png}

                      \centering

                      \begin{tabular}{cccc}

                        \ig[scale=0.3]{conenf.png} &   \ig[scale=0.3]{coneof.png} &  \ig[scale=0.3]{conend.png} &  \ig[scale=0.3]{coneod.png} \\

                        \ig[scale=0.2]{nein.png} &  \ig[scale=0.2]{checky.png} &  \ig[scale=0.2]{nein.png} &   \\

                        \tiny{Non straight non planar lines} &  \tiny{Straight planar lines} &  \tiny{Not  straight planar lines} &  \\

                      \end{tabular}


                  
                    \end{frame}


                    
                    \begin{frame}
                      

                      \ig[scale=0.85]{protoct2.png}

                      \centering

                      \begin{tabular}{cccc}

                        \ig[scale=0.3]{conenf.png} &   \ig[scale=0.3]{coneof.png} &  \ig[scale=0.3]{conend.png} &  \ig[scale=0.3]{coneod.png} \\

                        \ig[scale=0.2]{nein.png} &  \ig[scale=0.2]{checky.png} &  \ig[scale=0.2]{nein.png} & \ig[scale=0.2]{checky.png}  \\

                        \tiny{Non straight non planar lines} &  \tiny{Straight planar lines} &  \tiny{Not  straight planar lines} & \tiny{Straight non planar lines} \\

                      \end{tabular}

  
                    \end{frame}


                    \begin{frame}

                      \ig[scale=0.85]{protoct3.png}

                      \centering

                      \tiny{Test condition Sphere}

                      \ig[scale=0.8]{testy3.pdf}

                      \tiny{Test condition All surfaces}

                      \ig[scale=0.8]{surface.pdf}
                      
                      

                    \end{frame}


                    \subsection{Introspection measures}

                    \begin{frame}{Introspection measures }

                      \ig[scale=0.85]{introy.png}
                     

                      Confidence rating (CR) from 0 to 10

                      \centering
                      \ig[scale=0.17]{intro.png}

                      Insight reports ``Did you experienced an insight ?''

                      \tiny{(description adapted from Danek and Wiley 2017)}

                    \end{frame}
                


                    \section{Analyses}


                    \begin{frame}

                                        %% First, we tested whether participants’ objective performance varied as a function of the number of lessons studied. Observing effects of the teaching condition would show that our manipulation was successful in inducing learning, a sine-qua-non requisite for our study. Second, we looked for evidence that participants may have experienced insights, and tested whether these experiences were associated with a better level of objective understanding. Third and lastly, we tested whether insight reports were related to participant’s ratings of confidence, and analyzed how insight reports and ratings of confidence related to performance in the different subtests. Finding a double dissociation in this analysis would indicate that the mechanisms giving rise to insights are at least partially dissociated from the mechanisms informing people’s introspective judgments, supporting the hypothesis that some conceptual learning processes operate covertly and remain inaccessible to introspection.

                      
                      Does the paradigm allow participants to learn ?

                    
                      
                     \textcolor{white}{ Do participants presented with more lessons perform better in the post-teaching tests related to the new concept ?  }
                 
                      
                       
                    \end{frame}

      
                                        

                    \begin{frame}

                      Does the paradigm allow participants to learn ?                                         


                      Do participants presented with more lessons perform better in the post-teaching tests related to the new concept ?  

                    \end{frame}


                    \begin{frame}
                      
                     
                      
                      \centering

                      \ig[scale=0.4]{test1_cond.pdf}

                      \tiny{ \it Predictions of the logistic mixed model by test condition and number of lessons, with individual participants' performance, corrected for years of mathematic education after 10th grade} 
                      
                      \tiny{N=56, 18-43 years (\it M=25.5)}


                    \end{frame}
                    

                    \begin{frame}

                      \centering
                      \ig[scale=0.4]{test2_cond.pdf}

                      \tiny{N=56, 18-43 years (\it M=25.5)}
                    \end{frame}


                    \begin{frame}
                      
                      \centering
                      \ig[scale=0.4]{test3_cond.pdf}

                      \tiny{N=56, 18-43 years (\it M=25.5)}

                    \end{frame}

                    \begin{frame}
                                          
                      Do participants report insights ?
                      
                      Do insights seem to be triggered by the learning process ?

                                          
                    \end{frame}

                                        
                    
                    \begin{frame}
                                          

                      Do participants report insights ?


                    \end{frame}

                    \begin{frame}
                     

                      60 \% of insight reports

                      
                    \end{frame}


                    \begin{frame}
                      Do insights seem to be triggered by the learning process ?
                                          
                      \textcolor{white}{If learning a new concept triggers insight, reports of insight experiences should be more frequent when participants receive more lessons to study}


                      \end{frame}


                     \begin{frame}
                      Do insights seem to be triggered by the learning process ?
                                          
                     If learning a new concept triggers insight, reports of insight experiences should be more frequent when participants receive more lessons to study


                      \end{frame}
                   
                    \begin{frame}
                     



                      \centering

                       More insights reported in the conditions with more lessons (**)


                       \ig[scale=0.4]{ins_cond.png}

                      \tiny{Predictions of the binomial linear model by number of lessons, with individual participants' insight reports, corrected for years of mathematic education after 10th grade }

                      \tiny{N=56, 18-43 years (\it M=25.5)}
                      
                    \end{frame}
                    

                    \begin{frame}
                                          
                      Are insights related to a good performance ?

                      
                     \textcolor{white}{ If insight experiences reflect key computations in the underlying learning processes, those participants who report insights should achieve better levels of objective understanding, even when controlling for global confidence about understanding the new concept}

                      
                                          

                    \end{frame}


                    \begin{frame}
                                          
                      Are insights related to a good performance ?

                      
                      If insight experiences reflect key computations in the underlying learning processes, those participants who report insights should achieve better levels of objective understanding, even when controlling for global confidence about understanding the new concept

                      
                                          

                    \end{frame}


                      
                    
                    \begin{frame}

                      Are insights reports and confidence ratings dissociated ?

                      \end{frame}

                    
                    \begin{frame}


                      \ig[scale=0.65]{introy.png}
                      
                      \begin{table}
   
	                \label{tab:schemes}
	                \centering
	                \begin{tabular}{ccccc}
		          %% \toprule
                          \tiny Introspection measure	& \tiny CR 1		& \tiny CR 2		        & \tiny CR 3		& \tiny Insight\\

		          %% \midrule
	                  \tiny 	CR 1     	& \tiny X		& \tiny df(54)=.6 (<.0001)		& \tiny df(53)=.6 (<.0001)	& \tiny df(54)=.27 (.13) \\
	                  \tiny	CR 2    	& \tiny df(56)=.6 (<.0001)	& \tiny X     		        & \tiny df(53)=.85 (.<0001)	& \tiny df(54)=.19 (.33)\\
	                  \tiny	CR 3    	& \tiny df(55)=.6 (<.0001)	& \tiny df(55)=.85 (<.0001)		& \tiny X		& \tiny df(53)=.17 (.33) \\
	                  \tiny	Insight       	& \tiny df(56)=.15 (.81)	& \tiny df(56)=.14 (.81) 		& \tiny df(55)=.15 (.81)	& \tiny X          \\
		          %% \bottomrule
	                \end{tabular}
                      \end{table}

                      \tiny{Spearman's rho coefficients and p values for pairwise correlation tests. Above diagonal : zero-order correlation, below: with number of lessons and education in mathematics as covariates. All p values were corrected for multiple comparisons using Holm's  method}
                    \end{frame}
    



                    \begin{frame}

                      
                      Are insight reports uniquely related to a better performance in specific aspects of learning ?
                      
                    \end{frame}


                    \begin{frame}

                      \ig[scale=0.65]{protoct2.png}

                      
                      \centering


                     
                       \ig[scale=0.025]{cone_vd.JPG}

                      \ig[scale=0.27]{ins_difoui.pdf}  \ig[scale=0.27]{intro_difoui}
                      
                      
                      \tiny{Relation between performance and insight report or confidence ratings, in the condition that has no equivalent on the sphere (straight non-planar lines).
                        Predicted performance for participants who did vs. did not report insight experiences (resp. different levels of confidence), and individual participants’ performance corrected for years of mathematics education, number of lessons, and confidence level (resp. insight experiences)} 



                    \end{frame}

                    \section{Conclusion}
                    
                    \begin{frame}


                      
                      
                     Learning was effective and was function of the number of lessons 
                     



                     \textcolor{white}{ Two characteristic signatures of conceptual learning :}

                     \textcolor{white}{ Learning was difficult :  positive linear effects of the number of lessons on test performance}

                      %% These effects show that learning was not completed after studying the first lesson, and as all the lessons had the same mathematical content (great circles are straight, small circles are not straight), they show that participants benefited from repeated presentations of the same information. Strikingly, repeating information proved beneficial even in a test where participants simply had to recall the information presented in lessons, i.e. when judging that small circles drawn on spheres are not straight.

                     \textcolor{white}{ Content learned was inferentially rich: participants  were also able to draw inferences (better performance in generalization test conditions)}

                      


                    \end{frame}
                    
                    \begin{frame}


                      
                      
                     Learning was effective and was function of the number of lessons 
                     



                     Two characteristic signatures of conceptual learning :

                     \begin{itemize}

                       \item{\textcolor{white}{ Learning was difficult :  positive linear effects of the number of lessons on test performance}}

                      %% These effects show that learning was not completed after studying the first lesson, and as all the lessons had the same mathematical content (great circles are straight, small circles are not straight), they show that participants benefited from repeated presentations of the same information. Strikingly, repeating information proved beneficial even in a test where participants simply had to recall the information presented in lessons, i.e. when judging that small circles drawn on spheres are not straight.

                         \item{\textcolor{white}{ Content learned was inferentially rich: participants  were also able to draw inferences (better performance in generalization test conditions)}}

                      \end{itemize}


                    \end{frame}

                    \begin{frame}


                      
                      
                     Learning was effective and was function of the number of lessons 
                     



                     Two characteristic signatures of conceptual learning :

                     \begin{itemize}

                       \item{Learning was \textcolor{bittersweet}{difficult} :  positive linear effects of the number of lessons on test performance}

                      %% These effects show that learning was not completed after studying the first lesson, and as all the lessons had the same mathematical content (great circles are straight, small circles are not straight), they show that participants benefited from repeated presentations of the same information. Strikingly, repeating information proved beneficial even in a test where participants simply had to recall the information presented in lessons, i.e. when judging that small circles drawn on spheres are not straight.

                         \item{\textcolor{white}{ Content learned was inferentially rich: participants  were also able to draw inferences (better performance in generalization test conditions)}}

                      \end{itemize}


                    \end{frame}


                    \begin{frame}

                      

                      
                      
                     Learning was effective and was function of the number of lessons 
                     



                     Two characteristic signatures of conceptual learning :

                     \begin{itemize}

                       \item{Learning was \textcolor{bittersweet}{difficult} :  positive linear effects of the number of lessons on test performance}

                      %% These effects show that learning was not completed after studying the first lesson, and as all the lessons had the same mathematical content (great circles are straight, small circles are not straight), they show that participants benefited from repeated presentations of the same information. Strikingly, repeating information proved beneficial even in a test where participants simply had to recall the information presented in lessons, i.e. when judging that small circles drawn on spheres are not straight.

                   \item{   Content learned was \textcolor{bittersweet}{inferentially rich}: participants  were also able to draw inferences (better performance in generalization test conditions)}

                      \end{itemize}


                    \end{frame}

                    \begin{frame}
                      In the third test, teaching phase enabled participants (34\%) to draw non trivial inferences : \textcolor{bittersweet}{two straight lines drawn on a sphere can never be parallel, they necessarily cross}

                     \textcolor{white}{Counterintuitive for both Non/geometry-educated people \footnotesize{Izard et al. 2011}}

                      \normalsize

                      \textcolor{white}{ The experimental condition played a role in this understanding (linear trend of number of lessons for these two assertions, $\beta$ = 0.3, p = 0.001, controlling for years of education in mathematics)}


                    \end{frame}


                    \begin{frame}
                      In the third test, teaching phase enabled participants (34\%) to draw non trivial inferences : \textcolor{bittersweet}{two straight lines drawn on a sphere can never be parallel, they necessarily cross}

                      Counterintuitive for both Non/geometry-educated people  \footnotesize{Izard et al. 2011}

                      \normalsize

                     \textcolor{white}{The experimental condition played a role in this understanding (linear trend of number of lessons for these two assertions, $\beta$ = 0.3, p = 0.001, controlling for years of education in mathematics)}


                    \end{frame}

                    \begin{frame}
                      In the third test, teaching phase enabled participants (34\%) to draw non trivial inferences : \textcolor{bittersweet}{two straight lines drawn on a sphere can never be parallel, they necessarily cross}

                      Counterintuitive for both Non/geometry-educated people \footnotesize{Izard et al. 2011}

                      \normalsize

                      The experimental condition played a role in this understanding (linear trend of number of lessons for these two assertions, $\beta$ = 0.3, p = 0.001, controlling for years of education in mathematics)


                    \end{frame}


                    
                    \begin{frame}

                      
                      Learning a new concept gives rise to insight experiences (60\%)

                      \textcolor{white}{Extend the range of situations known to trigger insights \footnotesize{Bowden 2005, Webb 2016, Danek and Wiley 2016, Laukkonen 2020, Danek and Wiley 2014, Tian et al. 2017, Canestrari et al. 2017, Laukkonnen 2017}}




                    \end{frame}


                    \begin{frame}

                      
                      Learning a new concept gives rise to insight experiences (60\%)


                      Extend the range of situations known to trigger insights \footnotesize{Bowden 2005, Webb 2016, Danek and Wiley 2016, Laukkonen 2020, Danek and Wiley 2014, Tian et al. 2017, Canestrari et al. 2017, Laukkonnen 2017}




                    \end{frame}



                                        \begin{frame}

                      Insight experiences related to performance which was modulated across test conditions,
                      even when controlling for the number of lessons, math education, and confidence

                      \begin{itemize}

                       \item{ \textcolor{white}{Exclude several deflationary explanations : confabulation, artificial effect of experimental condition, overconfidence}}

                     \item{ \textcolor{white}{Reflect the functioning of learning processes as a term of a specific state, and the probability to reach this state increased when participants received more lessons to study}}
                       \end{itemize}

                    \end{frame}

                                        \begin{frame}

                      Insight experiences related to performance which was modulated across test conditions,
                      even when controlling for the number of lessons, math education, and confidence

                      \begin{itemize}

                       \item{ Exclude several deflationary explanations : confabulation, artificial effect of experimental condition, overconfidence}

                     \item{ \textcolor{white}{Reflect the functioning of learning processes as a term of a specific state, and the probability to reach this state increased when participants received more lessons to study}}
                       \end{itemize}

                    \end{frame}


                    \begin{frame}

                      Insight experiences related to performance which was modulated across test conditions,
                      even when controlling for the number of lessons, math education, and confidence

                      \begin{itemize}

                       \item{ Exclude several deflationary explanations : confabulation, artificial effect of experimental condition, overconfidence}

                     \item{ Reflect the functioning of learning processes as a term of a specific state, and the probability to reach this state increased when participants received more lessons to study}
                       \end{itemize}


                    \end{frame}




                     \begin{frame}
                      
                      Insights predicted performance in one test of generalization even after factoring out variations in ratings of confidence

                      \textcolor{white}{Learning abstract definitional properties of generalized straight lines involved processes that triggered  insights (thought did not inform participants’ introspective judgments of confidence)}

                      \begin{itemize}

                      \item{\textcolor{white}{Dissociated introspection processes }}

                      \item{\textcolor{white}{Insights could be necessary steps in concept learning }}

                      \end{itemize}


                    \end{frame}

                    \begin{frame}
                      
                      Insights predicted performance in one test of generalization even after factoring out variations in ratings of confidence

                      Learning abstract definitional properties of generalized straight lines involved processes that triggered  insights (thought did not inform participants’ introspective judgments of confidence)

                      \begin{itemize}

                      \item{\textcolor{white}{Dissociated introspection processes }}

                      \item{\textcolor{white}{Insights could be necessary steps in concept learning }}

                      \end{itemize}


                    \end{frame}

                    \begin{frame}
                      
                      Insights predicted performance in one test of generalization even after factoring out variations in ratings of confidence

                      Learning abstract definitional properties of generalized straight lines involved processes that triggered  insights (thought did not inform participants’ introspective judgments of confidence)

                      \begin{itemize}

                        \item{Dissociated introspection processes }

                        \item{\textcolor{white}{Insights could be necessary steps in concept learning }}

                          \end{itemize}


                      \end{frame}
                    


                    \begin{frame}
                      
                      Insights predicted performance in one test of generalization even after factoring out variations in ratings of confidence

                      Learning abstract definitional properties of generalized straight lines involved processes that triggered  insights (thought did not inform participants’ introspective judgments of confidence)

                      \begin{itemize}

                        \item{Dissociated introspection processes }

                        \item{Insights could be necessary steps in concept learning }

                          \end{itemize}


                      \end{frame}
                    
                    \section{Open questions}

                    
                    \begin{frame}

                      Are the insight experiences observed in the contexts of concept learning or problem solving qualitatively different, or do they reflect similar psychological processes?

                     \textcolor{white}{Second, if all insight experiences turn out to indicate the termination of a search process, what is the nature of the search involved in conceptual learning ?}

                     \textcolor{white}{Are insights experiences when learning other kinds of material, besides mathematical or science concepts ?}

                      \end{frame}



                    \begin{frame}

                      Are the insight experiences observed in the contexts of concept learning or problem solving qualitatively different, or do they reflect similar psychological processes?

                      If all insight experiences turn out to indicate the termination of a search process, what is the nature of the search involved in conceptual learning ?

                     \textcolor{white}{Are insights experiences when learning other kinds of material, besides mathematical or science concepts ?}

                      \end{frame}

                    \begin{frame}

                      Are the insight experiences observed in the contexts of concept learning or problem solving qualitatively different, or do they reflect similar psychological processes?

                      If all insight experiences turn out to indicate the termination of a search process, what is the nature of the search involved in conceptual learning ?

                      Are insights experienced when learning other kinds of material, besides mathematical or science concepts ?

                      \end{frame}

                    {
                      \setbeamercolor{background canvas}{bg=celadon}
                      \begin{frame}

                        \centering

                        Thanks for your attention !


                      \end{frame}
                    }
                    
\end{document}

